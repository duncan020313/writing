\documentclass[11pt, oneside]{article}    % use "amsart" instead of "article" for AMSLaTeX format
\usepackage{kotex}
\title{어떻게 표현할 것인가?}
\author{이동재}
\date{2024년 11월 28일}

\begin{document}
\maketitle

\begin{abstract}
신경망 기반 모델은 기존의 규칙 기반 모델로는 해결하기 어려웠던 이미지 분류, 언어 이해, 음성 인식 등 인지적 지능이 요구되는 다양한 문제를 효과적으로 해결해왔다. 그러나 이 과정에서 규칙 기반 모델의 장점이었던 명확하고 정확한 추론 능력을 잃게 되었다. 다음 세대 모델은 두 접근법의 장점을 결합하는 방향으로 나아가야 하며, 이를 위해서는 두 가지 다른 지식 표현을 연결할 수 있는 중간 표현이 필요하다. 오늘 글에서는 이러한 중간 표현의 필요성과 대략적인 형태에 대해 논할 것이다.
\end{abstract}

신경망 기반 모델은 가중치와 수치 벡터를 사용하여 지식을 표현한다. 이러한 전략은 인간의 인지 능력을 뛰어나게 모방하며, 기존의 명시적 규칙으로 설명하기 어려웠던 문제들에서 탁월한 성능을 보여준다. 하지만 이러한 수치 기반 표현은 엄밀한 추론을 수행하는 데에는 적합하지 않다.

반면, 규칙 기반 모델은 명시적인 규칙에 따라 길고 복잡한 논리적 추론을 성공적으로 수행할 수 있다. 하지만 인간의 직관이나 복잡한 인지적 문제를 다룰 때는 신경망 기반 모델과 비교하면 한계가 뚜렷하다. 사물이나 현상을 구분하는 기준이 명확하지 않아 규칙으로 기술하기 어렵기 때문이다.

신경망과 규칙 기반 모델은 상호보완적 장단점을 가지고 있다. 따라서 두 접근법의 장점을 결합하는 것은 필연적인 방향이다. 그리고 이를 완전히 구현하기 위해서는 서로 다른 지식 표현 방식인 수치 기반 표현과 논리적 표현을 융합할 수 있는 체계가 필요하다.

한 가지 접근법은 논리식에 수치 벡터를 통합하는 것이다. 신경망 기반 모델은 뛰어난 인지 능력을 갖추고 있기 때문에 논리식의 변수나 술어 (Predicate)를 수치 벡터로 표현하는 데 적합하다. 규칙 기반 모델은 이러한 수치 벡터를 변수로 활용해 기존의 논리적 추론을 수행할 수 있을 것이다.

또 다른 접근법은 신경망과 규칙 기반 모델이 공통으로 이해할 수 있는 중간 표현 (Intermediate Representation)을 정의하는 것이다. 신경망의 지식을 중간 표현으로 추출 (Distillation)하고, 규칙 기반 모델의 규칙을 중간 표현으로 컴파일함으로써 두 모델을 결합한 추론을 가능하게 할 수 있다.

차세대 모델의 지식 표현이 어떤 형태인지는 알 수 없으나, 그 역할은 분명하다. 이러한 발전 과정에 Synstiller 연구가 큰 기여를 할 수 있기를 바란다.
\end{document}