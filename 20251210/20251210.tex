\documentclass[11pt, oneside]{article}   	% use "amsart" instead of "article" for AMSLaTeX format
\usepackage{kotex}
\title{Can PL be Hero?}
\author{Dongjae Lee}
\date{2025.12.10}

\begin{document}
\maketitle


\begin{abstract}
  The conventional goal of the programming language is to make it easier to write correct and efficient programs.
  However, such goal is just a sidekick of the other fields like AI (e.g., building correct AI compiler).
  PL contributes a lot to the other fields, but it cannot be a hero by itself.
  Is this fundamental limitation of PL? Can we overcome this? In this essay, I will explore this question.
\end{abstract}

Programming language research has primarily served as a foundational tool for other domains. We build type systems to ensure the correctness of software systems in various fields. We design compilers to optimize the performance of artificial intelligence workloads. This supporting role, while vital, often keeps the field in the background.

To become a hero, a field must propose new problems rather than just solving existing ones. Leading disciplines define the agenda and steer the direction of technological progress. Currently, PL tends to react to challenges posed by hardware shifts or application demands. We need to shift from reactive problem-solving to proactive agenda-setting.

Programming languages can lead by redefining how humans interact with computation. We can create abstractions that make complex concepts accessible to non-experts. Such innovations would democratize software creation beyond the realm of professional engineers. By lowering the barrier to entry, PL can directly impact society at large.

Transforming PL into a protagonist requires a fundamental shift in our research mindset. We must look beyond correctness and efficiency to creativity and human empowerment. Only then can we move from being the unsung sidekick to the hero of the story.
\end{document}