\documentclass[11pt, oneside]{article}   	% use "amsart" instead of "article" for AMSLaTeX format
\usepackage{kotex}
\title{실험실과 현실 사이의 거리}
\author{Dongjae Lee}
\date{2025.08.29}

\begin{document}
\maketitle


\begin{abstract}
	실험실과 현실 사이의 거리는 굉장히 멀고 험난하다.
	실험실에서 아무리 좋은 결과를 도출했다 하더라도, 현실에서는 이를 활용하기 어려운 경우가 많다.
	특히, 실험실에서 현실로 가게 되면 신뢰도 혹은 수율 측면에서의 요구사항이 굉장히 까다로워 진다.
	AI 기술 역시 신뢰도 측면에서 현실의 요구치를 달성하지 못해 사람들에게 실망을 안겨준 경우가 많다.
	AI의 신뢰도 확보 기술은 앞으로 AI의 운명을 가르는 결정적 요소가 될 것이다.
\end{abstract}
지난 2025년 Security@KAIST에서 물방울의 곡률을 측정하여 가짜 분유를 가려내는 연구를 보았다.
굉장히 유용하고 많은 사람들에게 필요한 기술이지만, 정작 그 기술은 실험실 밖으로 나아가지 못했다.
연구자의 말을 들어보니, 식품과 관련된 기술은 식약처의 허가를 받아야 하는데, 요구되는 신뢰도가 99.99\% 이상이었기 때문에 포기하게 되었다고 한다.

이러한 문제는 비단 식품 분야 뿐만 아니라, 모든 연구와 현실 사이에 존재하는 문제이다.
소프트웨어 역시 다른 분야에 비해 그 정도가 덜 하긴 하지만 배포되는 소프트웨어는 좀 더 강한 테스트를 요구받는다.
AI 역시 실험실과 벤치마크 속에서는 날아다니지만, 정작 높은 신뢰도가 요구되는 분야에서는 여전히 도입 속도가 늦어지고 있다.

만약 AI의 신뢰도를 확보할 수 있는 좋은 기술이 탄생한다면 AI 기술은 지금보다 훨씬 빠른 속도로 세상 곳곳에 침투하게 될 것이다.
그 반대의 경우, AI 기술은 사람들의 기대에 부응하지 못하고 결국 또 다시 AI 겨울을 맞이하게 될 것이다.

속을 알 수 없는 소프트웨어의 안전성과 신뢰성을 확보하는 기술은 컴퓨터 공학 분야에서 아직 깊게 탐구한 적 없는 영역이다.
지금까지의 소프트웨어는 모두 소스코드로 작성된, 그 의미가 명확한 것들이었다.
그러나, 거대한 행렬로 구성된 신종 소프트웨어는 이해할 수도, 그 결과를 예측할 수도 없다.
이제는 이 새로운 소프트웨어를 어떤 관점에서, 어떻게 제어할 것인지에 대한 물음에 답해야 할 시점이다.

\end{document}