\documentclass[11pt, oneside]{article}   	% use "amsart" instead of "article" for AMSLaTeX format
\usepackage{kotex}
\title{인간과 닮아가는 AI, 우리가 원하는 AI인가?}
\author{이동재}
\date{}

\begin{document}
\maketitle


\begin{abstract}
  최근 AI분야에서는 텍스트를 넘어 사진을 비롯한 여러 데이터를 처리할 수 있는 멀티 모달 (Multi-Modal) 모델이 화제다. 텍스트 뿐만 아니라 모든 입력을 처리할 수 있기에 마치 일반 사용자의 입장에서는 마법처럼 느껴진다. 그러나, 이러한 발전이 진정으로 AI에게 기대하는 것일까? 이 글에서는 AI의 발전 방향에 대해 비판적인 시각을 제시한다.
\end{abstract}
지난 12일 OpenAI사의 GPT-4o 모델이 공개되었다 \footnote{OpenAI, Hello GPT-4o, 2024.05.13, https://openai.com/index/hello-gpt-4o/}. GPT-4o의 o는 접두사 `Omni'를 의미하며, 텍스트 뿐만 아니라 사진, 음성, 영상 자료를 모두 처리할 수 있는 멀티 모달 모델임을 나타낸다. 이는 AI가 더 이상 컴퓨터 속에만 존재하는 것이 아니라, 실제 세상과 상호작용 할 수 있는 능력을 갖추게 되었다는 것을 의미한다. 구글의 제미나이 (Gemini) 역시 이러한 기능을 갖춘 멀티 모달 모델이다\footnote{Google, Introducing Gemini: our largest and most capable AI model, 2023.12.06, https://blog.google/technology/ai/google-gemini-ai/}. 하지만, 과연 이러한 발전이 우리가 AI에게 기대하는 것일까?

많은 발전이 있었지만, 여전히 AI는 코드에서 오류를 발생시킨다. 보다 많은 종류의 데이터를 읽고, 보고, 이해할 수 있게 되었지만, 여전히 완벽하지는 않다. 인간의 활동에서 발생한 데이터로 학습했기 때문에, 결국 인간과 데이터 속 오류를 반영하게 된다. 이러한 한계는 데이터를 기반으로 학습하는 AI의 근본적 한계이며, 여전히 극복이 어려워 보인다.

인간이 컴퓨터에게 기대했던 것은 완벽하고, 실수를 발생시키지 않는 완벽한 계산 장치이다. 주어진 알고리즘을 정확하게 수행하고, 빠르게 결과를 도출해내는 것이 최우선 과제다. 그러나, AI는 인간과 같이 완벽하지 않고 실수를 하며 우리는 그 실수의 원인 조차 제대로 파악하지 못하고 있다.

실수와 같이 인간적인 면모를 해결하지 못한 AI라면 한계가 명확할 것이다. 여전히 신경망 기반의 AI는 법률, 의학, 항공 등 한 번의 실수가 치명적인 분야에서 인간의 역할을 완벽히 대체하지 못하고 있다. 뿐만 아니라 코드 작성 등 사무직에서도 여전히 AI는 인간의 승인을 받아야 한다. 정적 분석기가 인간의 코드를 승인해주는 것과는 대조되는 모습이다.

다음 단계로 나아가기 위해서 AI는 실수를 인지하고, 수정하는 능력을 갖추어야 한다. 그리고, 이러한 체계을 구축하기 위해서는 기호 기반 체계와의 협력이 필수적일 것이라 예상한다.
\end{document}