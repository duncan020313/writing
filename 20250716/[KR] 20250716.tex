\documentclass[11pt, oneside]{article}
\usepackage{kotex}
\title{프롬프트 작성 언어}
\author{Dongjae Lee}
\date{2025.07.16}
\begin{document}
\maketitle

\begin{abstract}
	ChatGPT의 아버지 Andrej Karpathy는 프롬프트는 새로운 형태의 프로그램이고, 언어 모델은 이를 실행하는 OS라고 한다.
	그런데, 프롬프트는 자연어로 작성되어 있어 일관된 내용을 말하는지, 문제가 있을 때 그 원인이 무엇인지 파악하기 매우 어렵다.
	자연어로 작성된 명세와 실제 구현의 불일치가 발생하는 것 처럼 프롬프트도 설명하려는 대상이 지속적으로 변화할 때 불일치가 발생할 가능성이 매우 높다.
	본 글에서는 이러한 문제의 원인과 해결 방안을 다룬다.
\end{abstract}

자연어로 작성된 프롬프트를 전부 읽고 기억하는 것은 매우 어렵다. 일례로, Claude의 시스템 프롬프트는 그 길이가 16,000단어에 달한다.
만약 프롬프트에서 설명하는 대상이 지속적으로 바뀌는 (e.g 프로그램, API 명세) 경우 프롬프트를 지속적으로 변경해주어야 하는데, 이는 매우 어려운 일이다.

이러한 문제는 전통적인 소프트웨어 명세와 소프트웨어 구현 사이의 불일치 문제와 매우 유사하다.
명세는 자연어로 작성되어 있기 때문에 책임자가 까먹는 경우 구현이 변경이 명세에 반영되지 않고, 이러한 불일치를 발견하는 것도 매우 어렵다.

프롬프트를 작성하는 프로그래밍 언어를 개발하는 것이 한 가지 해결 방법이 될 수 있다.
프롬프트가 형식언어로 작성된다면 이것의 올바름을 확인하는 검사, 검증 도구를 개발할 수 있기 때문이다.
그러나, 모든 내용을 형식언어로 작성하는 것은 강력하긴 하지만 일반인들에게는 매우 어렵고 비효율적이다.

또 다른 접근 방법은 자연어와 형식언어의 중간 형태를 찾고, 언어 모델을 활용하여 프롬프트의 올바름을 확인하는 것이다.
한 가지 예시로 변동이 있는 부분은 <Mutable> 태그로 감싸고, 언어 모델은 이 구역의 내용을 CI에서 반복적으로 확인하는 방식이다.
그 외에도 여러 가지 태그를 도입하여 일종의 프롬프트 작성용 Markup Language를 만들 수 있다.
\end{document}