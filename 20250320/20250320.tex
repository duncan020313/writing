\documentclass[11pt, oneside]{article}   	% use "amsart" instead of "article" for AMSLaTeX format
\usepackage{kotex}
\title{Reasoning Bandwidth}
\author{Dongjae Lee}
\date{2025.03.20}

\begin{document}
\maketitle


\begin{abstract}
  최근 
\end{abstract}
  Artificial General Intelligence (AGI) remains an inherently ambiguous and ill-defined concept. To date, there exists no official consensus on AGI, with researchers defining it according to their individual interpretations. However, a precise definition is essential to elevate AGI into the rigorous scientific discourse. How, then, can we define AGI with clarity and formally?

  In this article, I propose defining AGI as an artificial intelligence capable of solving all problems that humanity has solved thus far. For instance, such an AI must be able to solve all problems that were solved by Turing. A critical constraint in this framework concerns the training data: the AI can only learn knowledge that was available to Turing during his lifetime or knowledge from that era, and cannot access modern advancements.

  If such an artificial intelligence can successfully solve all problems addressed by eminent scientists such as Turing, Gauss, Einstein, and others, it could be classified as AGI—an intelligence capable of solving all problems within the scope of human intellectual achievement.

  The most challenging aspect of implementing this experimental approach lies in the impossibility of acquiring comprehensive training data from these renowned scientists. Furthermore, training data encompasses not only textual information but also images, audio, video, and other multimodal inputs. If we could simulate the virtual environments of these scientists' respective eras, we might be able to generate the necessary training data by reconstructing their historical contexts.
\end{document}