\documentclass[11pt, oneside]{article}   	% use "amsart" instead of "article" for AMSLaTeX format
\usepackage{kotex}
\title{한계를 넘어서}
\author{이동재}
\date{20240418}

\title{한계를 넘어서}
\author{이동재}
\date{20240418}

\begin{document}
\maketitle
\begin{abstract}
  나는 인공 지능이 인간의 인지 능력이 한계를 뛰어넘게 해줄 것으로 예측한다.
  이동 수단, 책과 문자, 전화와 인터넷은 운동 능력, 기억력, 정보 교환 능력의 측면에서 인간의 한계를 뛰어넘게 해주었다.
  인공 지능의 발전은 인간 인지 능력의 한계를 드러내고, 이를 극복하게 해줄 것이다.
  이미 바둑, 체스 등의 분야에서 인공 지능은 인간을 뛰어넘었지만, 해당 분야의 전문가들은 이에 좌절하지 않고 끊임없이 발전하고 있다.
  이러한 사례를 통해 인공 지능의 발전이 갖는 의미를 살펴보고자 한다.
\end{abstract}
최근 대형 언어 모델의 성능이 일부 영역에서 인간을 뛰어넘었고, 많은 사람이 두려움을 느끼고 있다.
``인공 지능이 인간의 지적 능력을 뛰어넘는다면, 인간은 어떤 가치를 지니는가?''
나는 이 질문에 이렇게 답하고 싶다. ``인공 지능의 발전은 곧 인간 지능의 발전으로 이어질 것이며, 인간 가치의 상실을 고민할 필요가 없다.''
글에서는 이 답변에 대한 근거를 알파고 (AlphaGo)의 탄생과 바둑의 관계를 통해 설명하고자 한다.

2016년 알파고의 등장은 바둑계의 입장에서 ChatGPT 이상의 충격이었다.
인간은 더는 바둑 인공 지능을 이길 수 없다. 당연한 수순으로 바둑과 바둑 기사의 가치에 대한 의문도 제기되었다.
많은 사람들은 바둑이 알파고의 등장으로 쇠락할 것이라 예상했다.

그러나, 바둑 인공 지능은 오히려 수십 년간 정체되어 있던 바둑의 발전을 촉진했다
\footnote{Henrik Karlsson, After AI beat them, professional Go players got better and more creative, https://www.henrikkarlsson.xyz/p/go}.
인공 지능의 수를 최고의 기사들이 모여 연구하고 분석하면서 새로운 바둑 기법이 탄생했다.
실제로, 바둑 기사들의 실력은 알파고와 릴라 제로 (Leela Zero) 등 바둑 인공 지능의 등장 이후 큰 폭으로 향상되었다.

한계를 마주해야만 그 너머로 갈 수 있다. 인공 지능의 발전은 인간의 한계를 드러내고, 그 한계를 넘어서는 계기가 될 것이다.
인간의 오랜 역사 동안 인간보다 똑똑한 존재는 없었다. 그렇기에 인공 지능의 발전이 두려움을 주는 것은 당연하다.
하지만 인공 지능의 발전으로 인간 지능의 장점과 부족함, 그 본질을 명확히 인지할 수 있을 것이다.
나아가 부족한 부분을 인공 지능을 통해 보완하고, 인간 지능의 한계를 넘어설 것이다.
\end{document}
