\documentclass[11pt, oneside]{article}
\usepackage{kotex}
\title{철근 콘크리트}
\author{Dongjae Lee}
\date{2025.08.05}
\begin{document}
\maketitle
\begin{abstract}
	철근 콘크리트는 철근과 콘크리트라는 상반된 성질의 건축 자재를 결합하여 두 자재의 장단점을 완벽히 상호 보완한다.
	이 소재의 개발로 인류는 엄청난 규모의 마천루를 건설할 수 있게 되었다.
	AI에도 논리 기반 AI와 신경망 기반 AI라는 완전히 다른 성질의 두 AI가 있다.
	결국 이 두 가지를 적절히 배합하는 것이 지능의 철근 콘크리트가 될 것이며, AGI 개발에 결정적인 역할을 할 것이다.
\end{abstract}

건축의 역사는 철근 콘크리트 개발 전후로 나뉠만큼 철근 콘크리트는 현대 건축에서 매우 중요한 자재다.
철근과 콘크리트는 각각 서로 다른 종류의 힘에 강한데, 두 가지를 섞으면 두 자재의 장점을 모두 살릴 수 있다.
게다가 두 재료는 거의 동일한 열팽창 계수를 가지고 있어 온도 변화가 있어도 균열이 발생하지 않는다.

AI 분야에서도 철근 콘크리트를 개발하려는 시도가 있다. 바로 뉴로-심볼릭 (Neuro-Symbolic) AI다.
신경망의 유연함과 논리의 안정성을 결합하여 두 AI의 장점을 모두 살리는 방식이다.

그러나, 안타깝게도 뉴로-심볼릭 AI는 아직 철근 콘크리트처럼 완벽한 배합 비율을 알아내지 못했다.
매우 제한적인 영역에서는 논리와 신경망을 어떤 시점에 써야 할지 명확하지만, 조금만 다루는 도메인이 커지면 그 경계가 모호해진다.
또한, 그 경계를 사람이 지정해주는 것 역시 매우 까다롭다.

기호와 신경의 적절한 배합 레시피만 알아낸다면 AI의 마천루를 세우는 것도 멀지 않을 것이다.
그 배합 레시피를 알아내는 사람이 다음 세대의 튜링상 수상자가 될 것이라 추측해본다.

\end{document}