\documentclass[11pt, oneside]{article}
\title{Reinforced Concrete}
\author{Dongjae Lee}
\date{2025.08.05}
\begin{document}
\maketitle
\begin{abstract}
	Reinforced concrete combines two building materials with opposing properties—steel rebar and concrete—to complement each other's strengths and weaknesses perfectly.
	The development of this material enabled humanity to construct enormous skyscrapers.
	In AI, there are also two distinct types of AI: symbolic AI and neural AI.
	Ultimately, properly mixing these two will become the reinforced concrete of AI and play a decisive role in the development of AGI.
\end{abstract}

The history of architecture can be divided into before and after the development of reinforced concrete.
Steel rebar and concrete each have strength against different types of forces: tension and compression.
When combined, both materials' advantages can be fully utilized.
Moreover, the two materials have nearly identical thermal expansion coefficients, preventing cracks from forming even with temperature changes.

There are also attempts to develop reinforced concrete in the AI field: Neuro-Symbolic AI. 
This approach merges the flexibility of neural networks with the stability of logic to harness the advantages of both AI types.

However, unfortunately, Neuro-Symbolic AI has not yet discovered the perfect mixing recipe like reinforced concrete.
In very limited domains, it's clear when to use logic versus neural networks.
But as soon as the domain becomes slightly larger, the boundary becomes ambiguous.
Furthermore, specifying that boundary manually is also extremely challenging.

If we can discover the proper mixing recipe for symbols and neural networks, building skyscrapers of intelligence won't be far off.
I speculate that the person who discovers this mixing recipe will be the next generation's Turing Award winner.

\end{document}