\documentclass[11pt, oneside]{article}   	% use "amsart" instead of "article" for AMSLaTeX format
\usepackage{kotex}
\title{Reflections on the Master's Degree}
\author{Dongjae Lee}
\date{2025.12.12}

\newcommand{\synstiller}{\textsc{Synstiller}}
\newcommand{\verisafeagent}{\textsc{VeriSafe Agent}}
\newcommand{\expecto}{\textsc{Expecto}}

\begin{document}
\maketitle


\begin{abstract}
  My master's course, which began in March 2024, has come to an end.
  It was a difficult journey, but isn't it true that we grow through hardship?
  In this article, I will share what I learned and felt by tracing the three research projects I pursued during the course.
\end{abstract}

My first project was \synstiller, which aimed to build an interpretable coding model.
Over nearly a year and a half, I invested substantial effort to solve the problem, but I did not achieve clear results.
However, knowing I did my best allowed me to let go with confidence, teaching me that doing my best is the key to moving on without regret.

After that, I worked on a project called \verisafeagent, which verifies an agent's behavior from a program-analysis perspective.
In this project, the biggest challenge was collaboration with another lab.
Aligning terminology and expressions, and writing the paper in the other field's language, were extremely demanding.
Through this experience, I realized that communication ability is a crucial skill for good research.

Finally, I worked on \expecto, which generates program specifications from natural language.
This was my first project in which I found, defined, and solved the problem entirely on my own from end to end.
Coincidentally, emerging discussions in the field validated that I was tackling a timely problem (e.g., spec-driven development).
This experience taught me that impactful research must go beyond personal curiosity to address actual real-world needs.

As I conclude my Master's degree, I am eager to apply the lessons I have learned to the next stage of my academic journey.
I would like to express my sincere gratitude to my advisor, labmates, parents, and friends for their unwavering support throughout this process.
\end{document}
