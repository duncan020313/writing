\documentclass[11pt, oneside]{article}   	% use "amsart" instead of "article" for AMSLaTeX format
\usepackage{kotex}
\title{명세가 온다}
\author{Dongjae Lee}
\date{}

\begin{document}
\maketitle


\begin{abstract}
	소프트웨어의 명세를 작성하는 것은 매우 중요한 일이지만, 등한시 되어온 면이 있다.
	명세를 유지보수 하는 것이 매우 까다로운 작업이기 때문이다.
	또한, 제품의 발전에 직접적인 영향을 주지도 않는다.
	그러나, 우리는 다시 한 번 명세에 주목해야 한다.
	AI 개발자의 등장은 프로그래머가 명세 작성자로 전환되는 결정적 계기가 될 것이다.
\end{abstract}

명세는 소프트웨어가 어떻게 동작해야 하는지, 어떤 기능을 수행하는지 담은 문서다.
즉, 실제 프로그램과는 독립적으로 존재하는 또 다른 구현체다.
명세를 활용하면 우리의 소프트웨어가 올바르게 작성되었는지 검증할 수 있고, 이는 소프트웨어의 안전성을 높여준다.

그러나, 명세를 유지보수하는 것은 매우 어렵고 귀찮은 작업이다.
명세가 직접적으로 프로그램의 성능이나 기능을 개선하는 것이 아니기 때문에 대부분의 회사에서는 이를 등한시한다.
한 번 뒤쳐진 명세는 아무도 사용하지도, 관리하지도 않는 레거시로 전락한다.

하지만, AI 개발자의 등장으로 상황은 완전히 뒤바뀌었다.
AI 개발자와의 소통에서 가장 중요한 것은 상세하고 구체적인 목표를 제시하는 것이다.
즉, 명세 작성이 가장 중요한 요소다.

이제는 명세를 어떻게 잘 작성할 수 있을지 고민해야 한다.
명세를 작성하는 방법과 도구는 아직 프로그래밍 그 자체에 비해서 덜 발전되어 있다.
앞으로의 프로그래밍 시스템 연구는 명세 작성과 그 관련 부분에 집중될 것으로 예상해본다.
\end{document}