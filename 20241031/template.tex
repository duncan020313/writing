\documentclass[11pt, oneside]{article}   	% use "amsart" instead of "article" for AMSLaTeX format
\usepackage{kotex}
\title{홉필드 네트워크}
\author{이동재}
\date{20241031}

\begin{document}
\maketitle


\begin{abstract}
  얼마 전 노벨 물리학상 수상자로 힌튼과 홉필드가 선정되면서 큰 화제가 되었다. 그들은 물리학적 법칙을 신경망에 적용하여 현대 생성 모델의 토대를 닦아 노벨상을 수상하게 되었다. 특히 홉필드는 기억이라는 행위를 물리학의 에너지 개념과 결합하여 홉필드 네트워크 (Hopfield Network)를 발명하였다. 본 글에서는 홉필드 네트워트가 어떻게 동작하는지 소개하고, 이러한 업적이 주는 교훈을 Synstiller와 연관지어 설명할 것이다.
\end{abstract}
  인간은 어떤 장면을 목격했을 때 그와 유사한 장면을 기억 속에서 즉각적으로 떠올리는 연상 기억 능력을 보유하고 있다. 만약 인간의 뇌가 데이터베이스처럼 동작한다면 모든 기억을 하나씩 훑으며 기존 기억화 현재 장면을 비교하느라 엄청난 시간을 필요로 할 것이다.
  
  홉필드는 이러한 기억 형성과 회상을 에너지 최적화 문제로 보았고, 이를 바탕으로 홉필드 네트워크를 발명하였다. 홉필드 네트워크는 정점과 양방향 간선으로 구성되며, 간선에는 가중치가 하나씩 할당되어 있다. 홉필드 네트워크의 학습은 곧 기억 형성과정으로 입력 데이터를 간선 가중치 속에 저장하는 과정이다. 네트워크에 에너지 개념을 도입하고, 입력 데이터가 낮은 에너지를 가지도록 가중치를 조정하는 방식으로 가중치에 기억을 새긴다.
  
  반대로 홉필드 네트워크의 추론은 기억 회상과정에 해당하며, 가중치를 고정하고 정점의 값을 바꾸며 에너지를 최소화한다. 학습 단계에서 입력 데이터가 낮은 에너지를 가지도록 했기 때문에 에너지가 낮아질수록 입력 데이터에 가까워진다. 고정점 (Fixed Point)에 도달하여 더 이상 에너지를 낮출 수 없으면 추론이 종료되고 그 시점의 정점 값이 추론 결과가 된다.

  홉필드 네트워크는 가중치에 기억을 자연스럽게 녹여냈으며, 이는 Synstiller에 적용될 수 있는 아이디어다. 현재 Synstiller는 특징을 추출하는 문맥 프로그램은 있지만, 기존에 봤던 모든 데이터를 벡터 데이터베이스로 모두 저장해두고 있다. 이로 인해 큰 데이터에 대해서는 많은 메모리와 시간을 요구한다. 만약 프로그램에 기억을 저장하고, 특징을 입력으로 제공했을 때 적절한 기억을 복원할 수 있다면 신경망 수준의 생성 능력을 확보하면서 해석 가능성을 잃지 않을 수 있는 길이 열릴 것이다. 
\end{document}