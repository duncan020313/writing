\documentclass[11pt, oneside]{article}   	% use "amsart" instead of "article" for AMSLaTeX format
\usepackage{kotex}
\title{환각의 원인}
\author{Dongjae Lee}
\date{2025.09.11}

\begin{document}
\maketitle


\begin{abstract}
	환각은 언어 모델의 가장 큰 문제점 중 하나이다.
	환각 현상으로 인해 모델을 신뢰할 수 없고, 낮은 신뢰도는 현실 세계에 언어 모델을 도입하는 것을 꺼리게 만든다.
	그렇다면 우리는 환각을 완전히 없앨 수 있을까?
	이번 글에서는 언어 모델이 환각을 일으키는 두 가지 원인에 대해 알아볼 것이다.
\end{abstract}
환각은 쉽게 말해 언어 모델이 없는 말을 그럴듯하게 지어내는 현상이다.
환각 현상은 언어 모델 초장기부터 약점으로 지적되었으나, 여전히 해결하지 못한 문제이다.
그 동안 검색 기반 생성 (Retrival Augmented Generation), 강화 학습 등 여러 기법이 제안되었으나, 완벽한 해결에는 실패했다.

그렇다면 언어 모델은 왜 환각을 일으킬까?
첫째로, 어떤 답도 100\% 확신할 수 없는 상황이 존재하기 때문이다.
주어진 정보만을 활용해서 문제의 답을 무조건 맞힐 수 없는 상황이 존재한다.
예를 들어, 무작위로 선택한 사람의 생일을 맞히는 것은 거의 불가능하다.

그렇다면, 모르는 상황에서는 모른다고 하면 되지 않을까?
언어 모델이 모른다고 하지 않는 이유는 찍는 것이 가만히 있는 것보다 낫기 때문이다.
우리가 시험 문제를 풀 때 모르는 문제를 보면 뭐라도 하나 찍는 것이 대부분이다.
가만히 있으면 기댓값이 0점이지만, 찍으면 기댓값이 1점이라도 올라가기 때문이다.
모델 역시 강화 학습 과정에서 정답은 1점, 오답은 0점 식의 이분법적 보상 함수를 사용하기 때문에 뭐라도 찍어 보는 행동이 강화되는 것이다.

결국 모든 것을 알 수는 없기에 모르는 상황에서는 모른다고 말할 줄 아는 `나 자신을 잘 아는' 모델이 필요하다.
나아가 내가 무엇을 모르는지 깨닫고, 이러한 정보를 스스로 탐색할 수 있다면 진정한 AGI가 아닐까?
\end{document}