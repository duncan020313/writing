\documentclass[11pt, oneside]{article}   	% use "amsart" instead of "article" for AMSLaTeX format
\usepackage{kotex}
\title{지식과 지능}
\author{이동재}
\date{20240808}

\begin{document}
\maketitle


\begin{abstract}
  지능이란 무엇인가? 어떻게 인공지능의 지능을 측정할 수 있는가? 우리는 반드시 이 물음에 답해야만 일반 인공지능에 대해 논할 수 있다. 그러나, 현재로써는 이러한 논의가 부족하다. 본 글에서는 이러한 문제를 지적하며, 인공지능이 지능을 가지고 있는지에 대한 의문을 제기한다.
\end{abstract}
최근 일반 인공지능으로의 도달 가능성에 대한 논쟁이 뜨겁다. 그러나 일반 인공지능으로의 도달 가능성에 앞서, 우리는 지능이 무엇인지 정의할 필요가 있다. 옥스퍼드 언어 사전에 따르면, `지능은 새로운 사물 현상에 부딪혀 그 의미를 이해하고 처리 방법을 알아내는 지적 활동의 능력'이라고 한다. 그렇다면, 인공지능은 정말 지능을 소유하고 있는 것일까?

언어 모델은 다양한 지식을 보유하고 있을 뿐 지능을 소유하고 있진 않다고 생각한다. 크게 두 가지 근거가 있다. 첫째로, 대형 언어 모델은 너무나 많은 데이터를 학습하여 더는 새로울 것이 없는 상태이다.
지능의 정의에 따라 지능을 측정하기 위해서는 `새로운 상황'에서의 이해 및 처리 능력을 보아야 한다. 학습 데이터의 크기가 벤치마크 데이터보다 압도적으로 거대하므로 올바른 방법으로 지능을 측정한다고 보기에는 무리가 있다.

역설적으로, 인공지능이 인간 수준의 지능을 갖추기 위해 인간보다 수천, 수만 배 많은 데이터가 필요하다는 것 역시 인공지능이 지능을 가지고 있지 않다는 증거이다.

둘째로, 언어 모델 지능의 결정적 증거 중 하나인 맥락 내 학습 역시 패턴 학습을 통해 충분히 구현할 수 있다. 우리가 고등수학을 공부하던 방식을 떠올려보자. 우리는 각 단원의 개념을 숙지한 후, 문제 유형에 관해 학습한다. 즉, 문제가 출제되는 패턴을 암기하는 것이다.
패턴을 암기하고, 그 패턴을 기반으로 문제를 푸는 행위는 지식을 활용한 문제풀이에 가깝다. 맥락 내 학습 역시 마찬가지다.

압도적으로 거대한 지식은 지식과 지능을 구분하는 경계를 흐리게 한다. 따라서 인간의 상식 수준에서 구축된 벤치마크를 통해서는 성능은 측정할 수 있어도 지능은 측정할 수 없다.
인공지능이 인간 수준의 지능을 지녔다 주장하기 위해서는 인공지능의 학습 데이터를 통제하여 제한된 지식 속에서 문제를 해결하는 능력을 입증해야 할 것이다.
\end{document}