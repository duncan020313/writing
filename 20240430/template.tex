\documentclass[11pt, oneside]{article}   	% use "amsart" instead of "article" for AMSLaTeX format
\usepackage{kotex}
\title{트랜스포머 (Transformer) 구조의 근본적 한계}
\author{이동재}
\date{}

\begin{document}
\maketitle


\begin{abstract}
  대형 언어 모델은 만능이 아니다. 트랜스포머 구조에 뿌리를 둔 대형 언어 모델이 놀라운 성과를 보이고 있지만, 특정 영역에서는 여전히 뚜렷한 한계를 보여준다. 이 글에서는 트랜스포머 구조가 지닌 근본적 한계와 뉴로-심볼릭 인공지능의 필요성에 대해 논의한다.
\end{abstract}
최신 대형 언어 모델은 대부분 트랜스포머 구조에 뿌리를 두고 있다. 트랜스포머 구조는 압도적인 문맥 파악 능력을 바탕으로 기존 RNN 기반의 구조를 완벽히 대체하였다. 자가 어텐션 (Self-Attention) 연산이 토큰 개수의 제곱에 비례하는 무거운 연산이지만, 하드웨어의 발전과 다양한 캐싱 전략의 개발로 이러한 단점을 어느 정도 극복하였다.

그러나, 트랜스포머 구조는 근본적으로 재귀 문제를 완벽하게 풀지 못한다. 즉, 트랜스포머 구조는 튜링 완전 (Turing complete)하지 않으며, 유한 상태 오토마타조차 완벽히 표현하지 못한다. 이러한 사실은 트랜스포머 구조가 패턴 인식, 자연어 언어 모델링, 기계 번역과 같은 특정한 문제에만 적합하다는 것을 의미한다.

최근 대형 언어 모델을 사용하여 수리, 논리 문제나 프로그래밍처럼 합성적 (Compositional)이고 재귀적인 영역의 문제를 해결하려는 시도가 있었다. 하지만, 문제의 길이가 매우 길어지는 경우, 문제 자체는 단순하더라도 트랜스포머 구조는 반드시 실패하게 된다. 패리티 검사 (Parity Check) 문제나 괄호 균형 문제와 같은 문제는 트랜스포머 구조로는 풀 수 없음이 이론과 실험을 통해 증명되었다.

트랜스포머의 한계를 극복하기 위해 최근에는 외부 메모리나 인공지능 에이전트 (Agent)의 도입을 통해 스스로 오류를 수정하고 개선하는 방향으로 연구가 진행되고 있다. 그러나, 이러한 방법들은 트랜스포머 구조의 근본적 한계를 해결하는 것이 아니라, 그 한계를 회피하고 지연시키는 방법에 불과하다.

이러한 관점에서 뉴로-심볼릭 (Neuro-symbolic) 인공지능의 도입은 필수적이다. 투명성과 정확성의 보장을 넘어 합성적이고 재귀적인 영역의 문제 해결을 위해서는 반드시 기호 기반의 추론 시스템이 필요하다. 상호 간의 단점을 보완하는 완전한 뉴로-심볼릭 인공지능이 다음 세대의 승자가 될 것이라 전망한다.
\end{document}