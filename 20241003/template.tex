\documentclass[11pt, oneside]{article}   	% use "amsart" instead of "article" for AMSLaTeX format
\usepackage{kotex}
\title{쉬운 문제부터 시작하기}
\author{이동재}
\date{20241003}

\begin{document}
\maketitle


\begin{abstract}
  생각의 사슬 (Chain of Thought, CoT)은 복잡한 문제를 단계적으로 해결하는 방법론으로, 언어 모델의 추론 성능을 개선하는 데 중요한 역할을 한다. 문제를 여러 개로 쪼개는 것도 중요하지만, 무엇을 먼저 풀지 결정하는 것도 매우 중요한 요소이다. 본 논문에서는 복잡한 문제에서 추론 순서의 중요성을 강조하고, 이를 고려한 언어 모델 학습 방법론을 제안한다.
\end{abstract}
  생각의 사슬은 언어 모델의 추론 성능을 개선하기 위한 중요한 언어모델 길들이기 기법 중 하나이다. 생각의 사슬은 '차근차근 생각하라'는 단순한 프롬프트로도 언어 모델이 복잡한 문제를 단계별로 해결하도록 유도할 수 있다. 이러한 방식은 추론 과정을 명시적으로 드러내기 때문에, 사용자에게도 보다 이해하기 쉬운 답변을 제공한다.

  최근 생각의 사슬은 단순히 문제를 순서대로 해결하는 것을 넘어서, 여러 갈래의 추론을 허용하거나 추론이 잘못되었을 때 이를 되돌릴 수 있는 보조 알고리즘을 도입하기도 한다. 이러한 방식은 모델이 더 많은 추론의 자유를 가지게 해주어, 성능을 향상시키는 데 기여한다.

  추론의 자유도 좋지만, 추론의 순서도 대단히 중요하다. 어려운 문제부터 먼저 풀려 한다면 차근차근 추론하는 의미가 없기 때문이다. 예를 들어, 곱셈 문제를 풀 때, 일의 자리 숫자부터 계산하는 것이 높은 자리 숫자를 먼저 계산하는 것보다 훨씬 더 간단하다. 높은 자리부터 계산하는 것은 결국 전체 계산을 한 번에 암산하는 것과 같기 때문이다. 그러나 언어 모델은 생성의 특성상 항상 앞부분부터 채워나가야 하기 때문에, 큰 숫자의 곱셈에서 어려움을 겪는다.

  따라서, 언어 모델이 복잡한 문제를 해결할 때, 쉬운 문제부터 순차적으로 해결하도록 훈련하는 것이 중요하다. 단순히 문제를 순차적으로 해결하도록 하는 것에 그치지 않고, '쉬운 문제부터' 해결하는 방식을 학습시키는 것이다. 이를 위해 강화 학습 알고리즘을 도입하면, 모델이 효율적인 추론 계획을 세우는 능력을 키울 수 있을 것이다.
\end{document}